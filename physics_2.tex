%%This is a very basic article template.
%%There is just one section and two subsections.
\documentclass{article}

%definitions for Formelsammlung

%\usepackage[left=1.5cm,right=1.5cm,top=2.5cm,bottom=2cm,landscape]{geometry} 
\usepackage[left=1cm,right=1cm,top=1cm,bottom=1cm,landscape]{geometry}
\usepackage{multicol}
\usepackage[ngerman]{babel}
\usepackage{tabularx}
\usepackage{mathpazo}
\usepackage{mathtools}
\usepackage{amsmath}  
\usepackage{setspace} 
\usepackage{commath}
\usepackage[utf8]{inputenc}
%\usepackage[ansinew]{inputenc}  
\usepackage[T1]{fontenc}
\usepackage{lmodern} 
\usepackage{hyperref}
\usepackage{bigints}
\usepackage{array}
\usepackage[table]{xcolor}
\usepackage{layouts}
\usepackage{siunitx}
\usepackage{wrapfig}
\usepackage{multirow,bigstrut}
\usepackage{trfsigns}
\usepackage{amssymb} 
\usepackage{fancyhdr}
\usepackage{datetime}
\usepackage{pgfplots}
\usepgfplotslibrary{fillbetween}
\usepackage{listings}
\usepackage{mathrsfs}
\usepackage{tabu}
\usepackage{pdflscape}
\usepackage{booktabs}
%\usepackage{mathabx}
\usepackage{graphicx}
\usepackage{supertabular}
\usepackage{siunitx}
\usepackage[europeanvoltages, europeancurrents, europeanresistors, americaninductors, smartlabels]{circuitikz}
\usepackage{xparse}
\usepackage{pdfpages}
\usepackage{bm}
\usepackage{braket}
\usepackage{relsize}
\usepackage{comment}
\usepackage{bm}

\DeclareMathOperator\arctanh{arctanh}
\DeclareMathOperator\arsinh{arsinh} 
\DeclareMathOperator\arcosh{arcosh}
\DeclareMathOperator\artanh{artanh}
\DeclareMathOperator\arcoth{arcoth} 
\DeclareMathOperator\sinc{sinc} 
\DeclareMathOperator\sgn{sgn} 
\DeclareMathOperator\LPF{LPF} 
\DeclareMathOperator\Q{Q} 
\DeclareMathOperator\erf{erf} 
\DeclareMathOperator\var{Var} 
\DeclareMathOperator\Cov{Cov} 
\DeclareMathOperator\floor{floor} 
\DeclareMathOperator\E{E} 
\DeclareMathOperator\NDFT{DFT} 
\DeclareMathOperator\IDFT{IDFT} 
\newcommand*\chem[1]{\ensuremath{\mathrm{#1}}}


%colorCodes
\definecolor{listinggray}{gray}{0.9}
\definecolor{lbcolor}{rgb}{0.95,0.95,0.95}
\definecolor{lightGray}{gray}{0.1}

\definecolor{cOrange}{HTML}{996633}
\definecolor{clOrange}{HTML}{DBB48D}
\definecolor{cBlue}{HTML}{336699}
\definecolor{clBlue}{HTML}{A0BCD8}
\definecolor{cGreen}{HTML}{339966}
\definecolor{clGreen}{HTML}{94D4B4}
\definecolor{cRed}{HTML}{993333}
\definecolor{clRed}{HTML}{D0B0B0}
\definecolor{cGray}{gray}{0.4}
\definecolor{clGray}{gray}{0.96}



\setlength{\parindent}{0pt}
%\DeclareMathOperator\arctanh{arccot}
\newcolumntype{L}[1]{>{\raggedright\let\newline\\\arraybackslash\hspace{0pt}}m{#1}}
\newcolumntype{C}[1]{>{\centering\let\newline\\\arraybackslash\hspace{0pt}}m{#1}}
\newcolumntype{R}[1]{>{\raggedleft\let\newline\\\arraybackslash\hspace{0pt}}m{#1}}
\newcolumntype{Y}{>{\centering\arraybackslash}X}
\newcolumntype{Z}{>{\raggedleft\arraybackslash}X}
\newcommand{\fmm}{\displaystyle} 
\newcommand{\cn}[1]{\underline{#1}} 
\newcommand{\hlaplace}{\quad\laplace\quad}
\newcommand{\hLaplace}{\quad\Laplace\quad}
\newcommand{\ztransform}{\, \, \xrightarrow{\, \, z\, \, } \, \,}
\newcommand{\zTransform}{\, \, \xrightarrow{\, \, z^{-1}\, \, } \, \, }
\newcommand{\infint}{\int_{-\infty}^{\infty}}
\newcommand{\infiint}{\iint_{-\infty}^{+\infty}}
\newcommand{\limint}{\lim_{T\rightarrow \infty} \frac{1}{T} \int_{-T/2}^{T/2}}
\newcommand{\bedeq}{\mathrel{\stackrel{\makebox[0pt]{\mbox{\normalfont\tiny WSS}}}{=\joinrel=}}}
\renewcommand{\fourier}{\mathcal{F}}
\newcommand{\infsum}[1]{\sum_{#1 = -\infty}^{\infty} }
\newcommand{\cif}{\text{if}\:}
\newcommand{\cand}{\:\text{and}\:}
\newcommand{\celse}{\text{otherwise}\:}
\renewcommand{\abs}[1]{\left| #1 \right|}
\newcommand{\cvec}[1]{\left[\begin{smallmatrix} #1 \end{smallmatrix}\right]}
\newcommand{\vvec}[1]{\renewcommand*{\arraystretch}{0.8}\left[\begin{array}{c} #1 \end{array}\right]}
\renewcommand{\hat}[1]{\widehat{#1}}
\let\oldsi\si
\renewcommand{\si}[1]{\; \left[\oldsi[per-mode = fraction]{#1}\right]}
%\newcommand*{\rom}[1]{\expandafter\@slowromancap\romannumeral #1@}
\newcommand{\rom}[1]{\textup{\uppercase\expandafter{\romannumeral#1}}}
\newcommand{\pdif}[2]{\frac{\partial #1}{\partial #2}}
\newcommand{\pdiff}[2]{\frac{\partial^2 #1}{\partial #2 ^2}}
\newcommand{\spinup}{\ket{\uparrow}}
\newcommand{\spindown}{\ket{\downarrow}}
\newcommand{\spinupup}{\ket{\uparrow \uparrow}}
\newcommand{\spinupdown}{\ket{\uparrow \downarrow}}
\newcommand{\spindownup}{\ket{\downarrow \uparrow}}
\newcommand{\spindowndown}{\ket{\downarrow \downarrow}}

\ExplSyntaxOn
\clist_new:N \l_feq_vector_clist
\NewDocumentCommand{\Vector}{O{\\}mO{b}}{
	\clist_set:Nn \l_feq_vector_clist {#2} % Set the list
	\renewcommand*{\arraystretch}{1}
	\begin{#3matrix}
		\clist_use:Nn \l_feq_vector_clist {#1} % show it with separator from #1 (\\)
	\end{#3matrix}
}
\ExplSyntaxOff

\newcommand{\plotTF}[1]{
\begin{tikzpicture}
\begin{axis}[xlabel=$\omega$,ylabel=$\abs{H(\omega}$, xmin = 0, xmax = 3.5, ymin = 0, ymax = 1, xtick = {3.14}, xticklabel={$\pi$}, ytick={0}, axis lines=middle, width=6cm, height=4cm, 
every axis x label/.style={at={(ticklabel* cs:1.05)},anchor=west}]
\addplot[name path=H, domain=0:3.14, samples=200] {#1};
\path[name path=axis] (axis cs:0,0.01) -- (axis cs:3.14,0.01);
\addplot[fill=clGray] fill between[of=H and axis, soft clip={domain=0.01:3.14}];
\end{axis}
\end{tikzpicture}
}

\newenvironment{donotbrake}{\begin{minipage}{\columnwidth}}{
\end{minipage} \vspace{1em}}

\newenvironment{cmat}[1]{
  \renewcommand*{\arraystretch}{0.9}
  \left[
  \begin{array}{#1}
}{
  \end{array}
  \right]
}

\newenvironment{case}{
  \left\{ \begin{array}{ll}
}{
  \end{array} \right.
}


\newenvironment{scase}{
  \renewcommand*{\arraystretch}{1}
  \left\{ \begin{array}{ll}
  }{
  \end{array} \right.
}

\newcommand\xdownarrow[1][2ex]{%
   \mathrel{\rotatebox{90}{$\xleftarrow{\rule{#1}{0pt}}$}}
}

\newcommand{\ncr}[2]{\binom{#1}{#2}}

\renewenvironment{description}{\color{cGray}}{}
\newenvironment{definition}{\color{cGray}}{} 
\newcommand{\cdef}[1]{\begin{definition}#1\end{definition}}


\newcommand{\vLaplace}[1][]{\mbox{\setlength{\unitlength}{0.1em}%
        \begin{picture}(10,20)%
          \put(3,2){\circle{4}}%
          \put(3,4){\line(0,1){12}}%
          \put(3,18){\circle*{4}}%
          \put(10,7){#1}
        \end{picture}%
       }%
 }%

\newcommand{\vlaplace}[1][]{\mbox{\setlength{\unitlength}{0.1em}%
        \begin{picture}(10,20)%
          \put(3,2){\circle*{4}}%
          \put(3,4){\line(0,1){12}}%
          \put(3,18){\circle{4}}%
          \put(10,7){#1}
        \end{picture}%
       }%
 }%         
 
           
\newenvironment{blockdiagram}[1]{
	\begin{tikzpicture}
		[auto, node distance=2.5cm,>=latex', scale=#1, every node/.style={scale=#1}]
}{
	\end{tikzpicture}
} 
 
 
\renewcommand{\arraystretch}{1.2}


\newenvironment{mtabular}[1] {
  \renewcommand{\arraystretch}{2}
  
  \begin{tabular}{#1}
}  
{
  \end{tabular}
}

\newenvironment{stabular}[1] {
\renewcommand{\arraystretch}{0.9}

\begin{tabular}{#1}
}  
{
\end{tabular}
}

\newenvironment{lmtabular}[1] {
\renewcommand{\arraystretch}{2}

\begin{supertabular}{#1}
}  
{
\end{supertabular}

\renewcommand{\arraystretch}{1.2}
}

\newenvironment{dtabular} {
  \begin{tabular}{>{\begin{definition}}l<{\end{definition}} >{\begin{definition}}l<{\end{definition}}}
}  
{
  \end{tabular}
}

\newenvironment{ddtabular} {
  \begin{center}
  \begin{tabular}{>{\begin{definition}}l<{\end{definition}} >{\begin{definition}}l<{\end{definition}} | >{\begin{definition}}l<{\end{definition}} >{\begin{definition}}l<{\end{definition}}}
}  
{
  \end{tabular}
  \end{center}
}


%configure tikz
%system description
\usetikzlibrary{shapes,arrows}
\usetikzlibrary{decorations.markings}
\usetikzlibrary{calc}
\tikzstyle{block} = [draw, rectangle, minimum height=2.5em, minimum width=5em]
\tikzstyle{input} = [coordinate]
\tikzstyle{output} = [coordinate]
\tikzstyle{pinstyle} = [pin edge={to-,thin,black}]
\tikzstyle{sum} = [draw, circle, node distance=1em, minimum height=1.5em]
\tikzset{
	>=latex,
	photon/.style={decorate,decoration={snake,post length=1mm, segment length = 2mm, amplitude=0.6mm}}
}
\tikzset{->-/.style={decoration={
			markings,
			mark=at position #1 with {\arrow{>}}},postaction={decorate}}}
\tikzset{%
  block/.style    = {draw, thick, rectangle, minimum height = 2.5em,
    minimum width = 3em},
  sum/.style      = {draw, circle, node distance = 1.5cm}, % Adder
  input/.style    = {coordinate}, % Input
  output/.style   = {coordinate}, % Output
  gain/.style     = {draw, thick, isosceles triangle, minimum height = 2em, isosceles triangle apex angle=60},
  rgain/.style    = {draw, thick, isosceles triangle, minimum height = 2em, isosceles triangle apex angle=60}
}


%externalize TIKZ
%\usetikzlibrary{external}
%\tikzexternalize[prefix=tikz/]

%lstlisting

\lstset{
  backgroundcolor=\color{lbcolor},
  tabsize=2,    
% rulecolor=,
  language=[GNU]C++,
  basicstyle=\scriptsize,
  upquote=true,
  aboveskip={1.5\baselineskip},
  columns=fixed,
  showstringspaces=false,
  extendedchars=false,
  breaklines=true,
  prebreak = \raisebox{0ex}[0ex][0ex]{\ensuremath{\hookleftarrow}},
  frame=single,
  numbers=none,
  showtabs=false,
  showspaces=false,
  showstringspaces=false,
  identifierstyle=\ttfamily,
  keywordstyle=\color{cBlue}
  commentstyle=\color{cGreen},
  stringstyle=\color{cRed},
  numberstyle=\color{black},
% \lstdefinestyle{C++}{language=C++,style=numbers}’.
}
\lstset{
  backgroundcolor=\color{lbcolor},
  tabsize=2,
  language=C++,
  captionpos=b,
  tabsize=3,
  frame=lines,
  numbers=none,
  numberstyle=\tiny,
  numbersep=5pt,
  breaklines=true,
  showstringspaces=false,
  basicstyle=\ttfamily,
  identifierstyle=\color{cOrange},
  keywordstyle=\color{cBlue},
  commentstyle=\color{cGreen},
  stringstyle=\color{cRed}
}

\lstdefinelanguage{makefile}{
  morekeywords={cc,CFLAGS,LFLAGS,OBJ,EXE},
  morecomment=[l]{\#}
}

\lstdefinestyle{makefile}{
  language=makefile,
  basicstyle=\ttfamily,
  keywordstyle=\color{cBlue},
  commentstyle=\color{cGreen},
  frame=lines,
  numbers=none,
  backgroundcolor=\color{lbcolor}
}


\newcolumntype{M}{>{$}c<{$}} % math-mode version of "c" column type

%header & footer
%\pagestyle{fancy}
%\lhead{Tibor Schneider}
%\rhead{Seite \thepage}
%\cfoot{\today} 

%\renewcommand{\headrulewidth}{0.4pt}
%\renewcommand{\footrulewidth}{0.4pt}

%Title of Document
\chead{Physics II - Summary} 

\begin{document}
\begin{twocolumn} 



\section{The Photon}  

\begin{donotbrake}
\begin{tabular}{cc}
	\begin{dtabular}
		
		$c \si{\metre \per \second}$ & speed of light \\
		$h \si{\metre \squared \kilogram \per \second}$ & planc's constant \\
		$e \si{\coulomb}$ & electorn charge \\
		$m_e \si{\kilogram}$ & electron mass \\
		$k_B \si{\metre \squared \kilogram \per \second \squared \per \kelvin}$ & bolzmann constant \\
		$\epsilon_0 \si{\farad \per \metre}$ & vacuum permittivity \\
		
	\end{dtabular}

	\begin{mtabular}{c}
		$c = 2.998 \cdot 10^8 \si{\metre \per \second}$ \\
		$h = 6.626 \cdot 10^{-34} \si{\metre \squared \kilogram \per \second}$ \\
		$\hslash = \frac{h}{2\pi}$ \\
		$e = 1.602 \cdot 10^{-19} \si{\coulomb}$ \\
		$m_e = 9.109 \cdot 10^{-31} \si{\kilogram}$ \\
		$k_B = 1.381 \cdot 10^{-23} \si{\metre \squared \kilogram \per \second \squared \per \kelvin}$ \\
		$\epsilon_0 = 8.854 \cdot 10^{-12} \si{\farad \per \meter}$ \\
		$1 \si{\electronvolt} = 1.602 \cdot 10^{-19} \si{\kilogram \metre \squared \per \second \squared} \si{\joule}$ \\
	\end{mtabular} 
\end{tabular}
\end{donotbrake}

\begin{donotbrake}
\subsection{Photon \& Electron}

\begin{tabular}{cc}
	\begin{dtabular}
		$\lambda \si{\meter}, \, \nu\si{\per\second}$ & Wavelength, Freq. \\
		$k$ & Wavenumber \\
		$E \si{\joule}$ & Energy \\
		$\vec{F_c} \si{\newton}$ & Coulomb Force \\
		%$\mu \si{\kilogram}$ & Equivalent Mass \\
	\end{dtabular} &
	\begin{mtabular}{c}
		$\fmm \lambda = \frac{c}{\nu} \quad \fmm \nu = \frac{c}{\lambda} \quad \omega = 2 \pi \nu$ \\
		$\fmm k = \frac{2 \pi \nu}{c}$ \\
		$E = h \cdot \nu = \hslash \cdot \omega$ \\
		$\fmm \abs{\vec{F_c}} = \frac{Q_1 \cdot Q_2}{4 \pi \epsilon_0 r^2}$ \\
		%$\mu = \frac{m_e m_p}{m_e+m_p}$\\
	\end{mtabular}
\end{tabular}

\end{donotbrake}


\begin{donotbrake}
\subsection{Photoelectric effect}

\begin{tabular}{cc}
	\begin{dtabular}
		$V \si{\volt}$ & Voltage \\
		$\phi_0 \si{\electronvolt} $ & Work function \\
		$I \si{\ampere}$ & Photo-current \\
		$n \si{\metre^{-3}}$ & Volume density of electrons \\
		$A \si{\metre \squared}$ & Area \\
		$v \si{\metre \per \second}$ & velocity of electrons \\	
	\end{dtabular} &
	\begin{mtabular}{c}
		$\fmm h \nu - \phi_0 = \frac{1}{2} m v^2 = eV$ \\
		$\fmm V(\nu) =  \frac{h}{e} \nu - \frac{\phi_0}{e}$ \\
		$\fmm I = n A v e$ \\
	\end{mtabular}
\end{tabular}
\end{donotbrake}


\begin{donotbrake}
\subsection{Blackbody Radiation}

\begin{ddtabular}
	$L \si{\metre}$ & length of blackbody cube &
	$k_i$ & wave constants \\
	$E_x$ & Electric field in x-direction &
	$<E>$ & Average Energy \\
	$N$ & Number of states &
	$D$ & Density of states \\
	$u$ & Blackbody radiation &
	$I$ & Power radiated \\
\end{ddtabular}

\begin{tabular}{c}
	$E_x(x,y,z) = E_{0x} \cos(k_x x) sin(k_y y) sin(k_z z)$ \\
	$\fmm k_x = n \frac{\pi}{L} \quad \fmm k_y = m \frac{\pi}{L} \quad \fmm k_z = l \frac{\pi}{L} \qquad k = \sqrt{k_x^2 + k_y^2 + k_z^2}$ \\
	$\fmm N(k) = \frac{1}{3\pi^2} k^3 L^3 \qquad D(k) = \frac{k^2}{\pi^2}$ \\
	$\fmm u(\omega) =\frac{\omega^2}{\pi^2 c^3} \cdot \frac{\hslash \omega}{\exp\left(\frac{-\hslash \omega}{k T}\right)-1} d\omega \qquad u(\nu) = \frac{8\pi h \nu^3}{c^3 \left( \exp \left(\frac{h \nu}{k T}\right) - 1\right)} d\nu$ \\
	$\fmm I(\omega) = c \cdot u(\omega)$
\end{tabular}

\textbf{Equipartition-Theorem}: Each degree of Freedom has an energy of $kT$
\end{donotbrake}

\begin{donotbrake}
\subsection{Johnson-Noise}

This is the noise created in a one-dimensional circuit (like a coax-cable).

\begin{tabular}{cc}
	
	\begin{tabular}{c}
	  \begin{circuitikz} [scale=0.6, transform shape]
	  	\draw [thick] (1,2) to [short] (5,2);
	  	\draw [thick] (1,0) to [short] (5,0);
	  	\draw (1,0) to [short, o-] (0,0) to [R=$R$] (0,2) to [short, -o ](1,2);
	  	\draw (5,0) to [short, o-] (6,0) to [R=$R$] (6,2) to [short, -o ](5,2);
	  	\draw (3,1.2) node {$Z_c = R$};
	  	\draw [latex-latex] (1,0.3) -- (5,0.3) node [pos=0.5, above] {$L$};
  	\end{circuitikz} \\ $\qquad$
	  \begin{circuitikz} [scale=0.6, transform shape]
	  	\draw (0,0) to [R=$R$] (0,2) to [sV_=$V$] (4,2) to [R=$R'$] (4,0) to [short] (0,0);
	  	\draw [dashed] (-0.8,-0.6) rectangle (2.6,2.6);
	  	\draw (-0.8,-0.6) node [below right]{Sample at Temperature $T$};
	  \end{circuitikz}
	\end{tabular} &

	\begin{tabular}{c}
		\begin{dtabular}
			$\langle V^2\rangle$ & Noise Voltage \\
			$\Delta \nu$ & Bandwidth \\
		\end{dtabular} \\
		\begin{tabular}{c}
			$E = E_0 \cdot \sin(k_x \cdot x)$ \\
			$\langle V^2 \rangle = 4 R \cdot k_BT \cdot \Delta \nu$
		\end{tabular}
	\end{tabular}
\end{tabular}
\end{donotbrake}

\begin{donotbrake}
\subsection{Momentum of a photon}

\begin{tabular}{cc}
	\begin{tabular}{c}
		\begin{tikzpicture}[scale=0.8, transform shape]
			\draw (2,0) rectangle (2.2,3);
			\draw [->, photon] (0,0.5) -- (2,1.5);
			\draw [->, photon] (1.9,1.6) -- (0,2.5);
			\draw [->] (2.2,1.5) -- (3,1.5) node[pos=0.5, above] {$v$};
		\end{tikzpicture}
	\end{tabular} &
	\begin{tabular}{c}
		\begin{dtabular}
			$p  \si{\kilo \gram \metre \per \second}$ &momentum \\
		\end{dtabular} \\
		\begin{tabular}{c}
			$\fmm p_{absorbing} =\frac{h \nu}{c}=m\cdot v$ \\
			$\fmm p_{reflecting} =2 \cdot \frac{h \nu}{c}$\\
			$p=\sqrt{2m_ee\Delta V}$
		\end{tabular}
	\end{tabular}
\end{tabular}
\end{donotbrake}

\begin{donotbrake}
\subsection{Absorption, spontaneous and stimulated emission}
\begin{center}
\begin{tabular}{ccc}
	\begin{tikzpicture}
		\draw[thick] (0,0) -- (1,0) (0,2) -- (1,2);
		\draw[*->] (0.5,0) -- (0.5,2) node[pos=0.5, right]{$B_{12}$};
		\draw[->,photon] (-0.5,1) -- (0.5,1);
	\end{tikzpicture} &
	\begin{tikzpicture}
	\draw[thick] (0,0) -- (1,0) (0,2) -- (1,2);
	\draw[<-*] (0.5,0) -- (0.5,2) node[pos=0.5, left]{$A_{21}$};
	\draw[->,photon] (0.5,1) -- (1.5,1);
	\end{tikzpicture} &
	\begin{tikzpicture}
	\draw[thick] (0,0) -- (1,0) (0,2) -- (1,2);
	\draw[<-*] (0.5,0) -- (0.5,2) node[pos=0.8, right]{$B_{21}$};
	\draw[->,photon] (-0.5,1) -- (0.5,1);
	\draw[->,photon] (0.5,1.2) -- (1.5,1.2);
	\draw[->,photon] (0.5,0.8) -- (1.5,0.8);
	\end{tikzpicture} \\
	absorbtion & spontaneous emission & stimulated emission
\end{tabular}
\end{center}

\begin{center}
	\begin{dtabular}
		$n_1$ & Number of electrons in the lower energy state \\
		$n_2$ & Number of electrons in the higher energy state \\
	\end{dtabular} 
\end{center}

$$\fmm \frac{dn_2}{dt} = \underbrace{n_1 \cdot u(\nu) \cdot B_{12}}_\text{absorbtion} - \underbrace{n_2 \cdot u(\nu) \cdot B_{21}}_\text{stimulated emission} - \underbrace{n_2 \cdot A_{21}}_\text{spontaneous emission} $$
$$\fmm \frac{n_2}{n_1} = e^{-\frac{h \nu}{k_B T}} = \frac{u(\nu) B_{12}}{u(\nu) B_{21} + A_{21}}$$
$$\fmm B_{21} = B_{12} = B \qquad A_{21} = \frac{8\pi h \nu^3}{c^3}$$

\end{donotbrake}

\subsection{Laser-optical amplification}

\begin{center}
	\begin{tabular}{cc}
		\begin{tikzpicture}
		\draw [-implies, double equal sign distance] (0.3,0.5) -- (0.3,3.5) node[pos=0.5, left] {R};
		\draw [->, dashed] (0.7,3.5) -- (1.5,3);
		\draw [->, dashed] (1.5,1) -- (0.7,0.5);
		\draw [->] (1.5,3) -- (1.5,1);
		\draw [thick] (0,0.5) -- (1,0.5) node[right]{0} (0,3.5) -- (1,3.5) node[right] {3} (1,3) -- (2,3) node[right] {2} (1,1) -- (2,1) node[right] {1};
		\end{tikzpicture} &
		\begin{tikzpicture}
		\draw [fill=black!5] (1,0) -- (1,2) -- (5,2) node[pos=0.5, above] {amplification medium} -- (5,0) -- (1,0);
		\draw [thick] (0,0) arc (-150:-210:2);
		\draw [thick] (6,0) arc (-30:30:2);
		\draw [->-=.35] (-0.1,1.8) arc ( 250: 290:9.05);
		\draw [->-=.75] (-0.1,1.8) arc ( 250: 290:9.05);
		\draw [->-=.75] ( 6.1,1.8) arc ( -70: -110:9.05);
		\draw [->-=.35] ( 6.1,1.8) arc ( -70: -110:9.05);
		\draw [->-=.35] (-0.1,0.2) arc (-250:-290:9.05);
		\draw [->-=.75] (-0.1,0.2) arc (-250:-290:9.05);
		\draw [->-=.75] ( 6.1,0.2) arc (70:110:9.05);
		\draw [->-=.35] ( 6.1,0.2) arc (70:110:9.05);
		\draw [implies-,double equal sign distance] (3,0) -- (3,-0.5) node[below] {pump};
		\draw [->, dashed] (6.3,1) -- (7.3,1);
		\draw [->, dashed] (6.2,1.5) -- (7.3,1.5);
		\draw [->, dashed] (6.2,0.5) -- (7.3,0.5);
		\draw (6,2) node[above] {cavity};
		\end{tikzpicture}
	\end{tabular}
\end{center}


Electrons are excited from the ground state ``0'' to the level ``3'' by pumping through incoherent radiation. 
The electrons then fall onto a long-lived state $n_2$ (State ``2'') from level ``3''. 
The pumping can be done either optically by shining a strong incoherent light or by passing a current. 
It is also assumed that the lower state is quickly emptied by a fast process with lifetime $\tau_1$. 
As a result, the population in state ``2'' is:
$$n_2 = \frac{R}{A_{21}} \quad \text{whereas} \quad n_1 \approx 0 \quad \text{because} \quad  A_{21} < \frac{1}{\tau_1}$$

\begin{comment}
We have rherefore a population inversion between the two states. 
The likelihood of a stimulated emission process is larger than the one of absorbtion. 
If we enclose the system in an optical cavity, we can achieve self-sustained oscillation at the frequency $\nu$.

\end{comment}

\subsection{Fermi Energy of a metal}

$$E_F = \frac{\hslash^2}{2m_e} \left( 3 n \pi^2 \right)^{2/3} \qquad n = \frac{\rho}{m} = \frac{\rho \cdot N_A}{m_{mol}}$$

Where $m_e \si{\kilogram}$ is the mass of the electron, $m \si{\kilogram}$ is the mass of an single atom of the metal, $m_{mol} \si{\kilogram \per \mole}$ is the atomic weight, $n \si{\per \meter \cubed}$ is the number of atoms per unit of volume and $\rho \si{\kilogram \per \metre \cubed}$ is the density of the metal. 

\section{Wave mechanics}

\begin{center}
	\begin{tabular}{ccccc}
		& frequency & wavelength & momentum & energy \\
		\midrule
		Particle & & $\lambda_b = \frac{h}{p}$ & $p = m v$ & $E = \frac{1}{2} m v^2$ \\
		Wave & $\omega$ & $\lambda = \frac{2\pi c}{\omega}$ & $p = \frac{\hslash \omega}{c}$ & $E = \hslash \omega$ \\
	\end{tabular}
\end{center}

\subsection{Compton Scattering}

\begin{tabular}{cc}
	\begin{tabular}{c}
		\begin{tikzpicture}
			\draw [->, photon] (0,0) -- (1.3,0) node[pos=0.5, below] {$p_1$};
			\draw [fill=black] (2,-0.1) circle (0.08) node[below] {$e^-$};
			\draw [->, photon] (4,0) -- (5.5,1) node[pos=0.5, above left]{$p_2$};
			\draw [fill=black] (4,0) circle (0.08) node[below] {$e^-$};
			\draw [->] (4,0) -- (5.5,-1.3) node[pos=0.7, below] {$v$};
			\draw [dashed] (4,0) -- (5.5,0);
			\draw (4.8,0) arc (0:33:0.8) node[pos=0.6, right] {$\theta$};
			\draw (4.9,0) arc (0:-41:0.9) node[pos=0.6, right] {$\phi$};
		\end{tikzpicture}
	\end{tabular} &
	\begin{mtabular}{c}
		$\fmm p_1 =\frac{h \nu_1}{c} \qquad p' = \frac{h \nu_2}{c}$ \\
		$\fmm \nu_2 = \nu_1 - \frac{P_e^2}{2 m_e h}$ \\
		$\fmm \lambda_2 - \lambda_1 = \frac{h}{m_e c} (1 - \cos \theta)$;
	\end{mtabular}
\end{tabular}

\subsection{Double Slit and Bragg Diffraction}

\begin{tabular}{ccc}
	\begin{tabular}{c}
		\begin{tikzpicture} [scale=0.8, transform shape]
		\draw [fill=black!10] (-0.05,0.05) rectangle (0.05,0.85) (-0.05,0.95) rectangle (0.05,1.45) (-0.05,1.55) rectangle (0.05,2.35);
		\draw [|-|] (-0.2,1.5) -- (-0.2,0.9) node [pos=0.5,left] {\small $a$};
		\draw [dashed] (0,1.2) -- (2,1.2);
		\draw (0,1.2) -- (2,0.6);
		\draw (1.5,1.2) arc (0:-17:1.5);
		\draw (1.2,1) node {\small $\theta$}; 
		\end{tikzpicture}
	\end{tabular} &
	
	\begin{tabular}{c}
		\begin{tikzpicture} [scale=0.8, transform shape]
		\foreach \y in {0,0.4,0.8,1.2,1.6,2.0} {
			\draw [fill=black!10] (-0.05,\y+0.05) rectangle (0.05,\y+0.35);
		}
		\draw [|-|] (0.2,0.8) -- (0.2,0.4) node[pos=0.5, right] {\small $a$};
		\draw [dashed] (0,1.6) -- (2,1.6);
		\draw (0,1.6) -- (2,1);
		\draw (1.5,1.6) arc (0:-17:1.5);
		\draw (1.2,1.4) node {\small $\theta$}; 
		\end{tikzpicture}
	\end{tabular} &
	\begin{mtabular}{cl}
		Constructive & $\fmm \sin \theta = \frac{n \lambda}{a}$ \\
		Destructive &$\fmm \sin \theta = \frac{(n +\frac{1}{2}) \lambda}{a} $ \\
	 & $n \in \mathbb{Z}$
	\end{mtabular}
\end{tabular}

\subsection{Single slit and uncertainty relation}

\begin{tabular}{cc}
	\begin{tabular}{c}
		\begin{tikzpicture}[scale=0.7, transform shape]
			\draw [->, photon] (-1.5,1) -- (-0.7,1);
			\draw [thick] (0,-0.5) -- (0,0.7) -- (0.1,0.7) -- (0.1,-0.5);
			\draw [draw=none, fill=black!10] (0,-0.5) rectangle (0.1,0.7); 
			\draw [thick] (0,2.5) -- (0,1.3) -- (0.1,1.3) -- (0.1,2.5);
			\draw [draw=none, fill=black!10] (0,2.5) rectangle (0.1,1.3);
			\draw [|-|] (-0.2,0.7) -- (-0.2,1.3) node[pos=0.5, left] {$a$};
			\draw (2,-0.5) -- (2,2.5);
			\draw [thick, domain=-1.51:1.5, samples=150, variable=\y] plot ({30*sin(\y*200)*sin(\y*200)/((\y*20)*(\y*20)) + 2},{\y + 1}); 
			\draw (2.5,1.8) node {$I(\theta)$};
			\draw [dashed] (0,1) -- (2,1);
			\draw (0,1) -- (1.95,1.75);
			\draw (1.5,1) arc (0:21:1.5) node[pos=0.5, left] {$\theta$};
		\end{tikzpicture}
	\end{tabular} &
	\begin{mtabular}{c}
	$\fmm I(\theta) = I_0 \frac{\sin^2 \theta}{\theta^2}$ \hspace{2cm}
		$\fmm \sin \theta = \frac{\lambda}{a}$\\
		  $\fmm\Delta x\Delta p \geq \hslash$ \hspace{1cm} $\fmm\Delta t\Delta E \geq \hslash$ $(E=\hslash\omega)$
		
	\end{mtabular}
\end{tabular}

\subsection{Bohr-Sommerfeld quantisation}

Every single particle must satisfy the following equation. The quantized energy levels below relate to the hydrogen atom

\begin{tabular}{cc}
	\begin{dtabular}
		$p$ & Momentum of particle \\
		$E_n$ & Energy of the nth state \\
		$E_{ry}$ & Rydberg Energy \\
		$a_0$ & Bohr-radius \\
		$Z$ & Number of protons \\	
	\end{dtabular} &
	\begin{mtabular}{c}
		%$2\pi r = n\lambda$\\
		$\fmm \int_{length} p \cdot ds = n \cdot h \qquad n \in \mathbb{N}$ \\
		$\fmm E_n = -\frac{Z^2}{n^2} \cdot \frac{m_e e^4}{8 \epsilon_0^2 h^2} = - \frac{Z^2}{n^2} \cdot E_{ry} $ \\
		$\fmm r_n = \frac{n^2}{Z} \cdot \frac{2 \epsilon_0 h}{m_e e^2} = \frac{n^2}{Z} \cdot a_0$ \\
		$E_{ry} = 13.6 \si{\electronvolt}$ \\
		$a_0 = 5.292 \cdot 10^{-11} \si{\meter}$
	\end{mtabular}
\end{tabular}

\section{Quantum Mechanics}

\subsection{Wave function}
$$\psi(\bm{x}, t): \mathbb{R}^4 \rightarrow \mathbb{C} \qquad \iiint \abs{\psi(\bm{x},t)}^2 d^3r = 1 $$
$$\psi(\bm{x}, t) = a \psi_1(\bm{x},t) + b \psi_2(\bm{x}, t), \qquad \abs{a}^2 + \abs{b}^2 = 1$$
$$P(x)dx=\abs{\psi (x)}^2dx \qquad \scriptstyle{P_{ab}=\int_{a}^{b}\abs{\psi (x)}^2dx \qquad \langle x\rangle=\int_{-\infty}^{\infty}x\abs{\psi (x)}^2dx}$$

\subsection{The Schrödinger equation}
%\begin{ddtabular}
%	$V(x,t)$ & potential &
%	$m$ & mass \\
%\end{ddtabular}
$$i \hslash \cdot \frac{\partial \Psi}{\partial t}(\bm{x},t) = - \frac{\hslash^2}{2m} \cdot \nabla^2 \Psi(\bm{x},t) + V(\bm{x}, t) \Psi(\bm{x},t)$$
$$\Psi = A \cdot e^{i (\bm{k} \bm{x} - \omega t)} \qquad \bm{k} = \Vector[&]{k_x,k_y,k_z}, \quad \bm{x} =\Vector[&]{x,y,z}^T$$
$$E = \omega \hslash = \frac{\hslash^2 k^2}{2 m}, \qquad k^2 = \abs{k}^2$$

The wave function and it's derivative must be continuous where the potential $V(x,t)$ is finite.

\subsubsection{Phase and Group Velocity}
\cdef{phase velocity $v_{\varphi}$} (phase movement), \cdef{group velocity $v_g$} (movement of wave packet)
$$v_{\varphi} = \frac{\omega}{k} \qquad v_g =\frac{\partial \omega}{\partial k} \qquad \text{Particle wave: } \ v_\varphi \cdot 2 = v_g$$

\subsubsection{Stationary (Time independent) States}

In a stationary state, the wave function is a product of a function $\varphi(\bm{x})$ independent of time and a function $\chi(t)$ independent of space. 

$$\Psi_n(\bm{x},t) = \psi_n(\bm{x}) \cdot \chi_n(t) = \psi_n(\bm{x}) \cdot e^{-i \frac{E_n}{\hslash}t}$$
$$-\frac{\hslash^2}{2m} \nabla^2 \psi_n(\bm{x}) + V(\bm{x}) \psi_n(\bm{x}) =\psi_n(\bm{x}) \cdot E_n$$
$$\iiint \abs{\Psi}^2 d^3\bm{x} = \iiint \abs{\psi}^2 d^3\bm{x} = 1$$
$$\Psi(\bm{x},t) = \sum a_n \psi_n(\bm{x}) \cdot e^{-i \frac{E_n}{\hslash}t} \quad \sum \abs{a_n}^2 = 1$$

\subsubsection{Example: 1D infinite potential well}

\begin{tabular}{cc}
	\begin{tabular}{l}
		\begin{tikzpicture}
			\draw [fill=black!10, draw=none] (0,0) rectangle (1,2) (3,0) rectangle(4,2);
			\draw [->] (0,0) -- (4.1,0) node[right] {$x$};
			\draw [->] (1,0) -- (1,2.1) node[below right] {$V(x)$};
			\draw (3,0) -- (3,2);
			\draw (0.5,1) node {1} (2,1) node {2} (3.5,1) node {3};
			\draw (1,0) node[below] {0} (3,0) node[below] {$L$};
		\end{tikzpicture}
	\end{tabular} &
  \begin{mtabular}{l}
  	$\fmm \Psi_{1} =\Psi_{3} = 0$ \\
  	$\fmm -\frac{\hslash^3}{2m} \frac{\partial^2}{\partial x^2} \psi_2(x,t) = E \psi_2(x,t)$ \\
  	$\fmm \psi_2 = A \sin(k x) + B \cos(k x)$ \\
  	Boundary cond.: $\fmm \psi_2(0) = \psi_2(L) = 0$ \\
  \end{mtabular}
\end{tabular}

\begin{mtabular}{c}
	$\fmm \psi_{2_n} = A \cdot \sin (k_n x) \quad \Psi_{2_n} = A \cdot \sin \left( k_n x \right)  \cdot e^{-i \frac{E_n}{\hslash} x}, \quad \text{\small Normalize:} \quad  A = \sqrt{\frac{2}{L}}$ \\
	$\fmm E_n =n^2 \cdot \frac{\hslash^2 \pi^2}{2 m L} =n^2 \cdot E_0, \qquad k_n = \frac{n \pi}{L} \si{\per \meter}$ \\
\end{mtabular}

\subsubsection{Example: 1D finite potential well}

\begin{tabular}{cc}
	\begin{tabular}{l}
		\begin{tikzpicture}
			\draw [->] (0,1.5) -- (4,1.5) node[right] {$x$};
			\draw [->] (2,0) -- (2,2) node[above] {$V(x)$};
			\draw [thick, cRed] (0,1.5) -- (1,1.5) -- (1,0.25) -- (3,0.25) -- (3,1.5) -- (3.9,1.5);
			\draw (0.5,1) node {1} (2,1) node[right] {2} (3.5,1) node {3};
			\draw (1,1.5) node[above] {$-L$} (3,1.5) node[above] {$L$};
			\draw (2,0.25) node[above left] {$-V_0$};
		\end{tikzpicture}
	\end{tabular} &
	\begin{tabular}{p{0.5\columnwidth}}
		The Energy $E$ can be either bigger or smaller than 0. If $E>0$, the wave function will decay exponentially in region 1 and 3. If $E < 0$, the wave will propagate away from the potential well.
	\end{tabular}
\end{tabular}

\textbf{Inside the well: }The general solution to the rearranged Schrödinger's is:

$$-\frac{\hslash^2}{2m} \frac{\partial^2}{\partial x^2}\psi_2(x) = (E-V_0) \psi_2(x)$$
$$\psi_2(x) = A_2 e^{i k x} + A'_2 e^{-i k x} \qquad E = \frac{k^2 \hslash^2}{2m} \quad k = \sqrt{\frac{2 m (E-V_0)}{\hslash^2}}$$

\textbf{Outside the well: }There are two cases, which can apply:

\begin{enumerate}
	\item $E > 0$:\textbf{Unbound state}
	$$-\frac{\hslash}{2m} \frac{\partial^2}{\partial x^2}\psi_{1}(x) = E \psi_{1}(x) \qquad \psi_1 =  A_1 e^{i k x} + A'_1 e^{-i k x} \qquad k = \sqrt{\frac{2 m E}{\hslash^2}}$$
	The unbound state does not make sense to be investigated, because the particle is free to be anywhere. In the following, only the unbound state is considered.
	\item $E < 0$: \textbf{Bound state}
	$$-\frac{\hslash}{2m} \frac{\partial^2}{\partial x^2}\psi_{1}(x) = E \psi_{1}(x) \qquad \psi_1 = B_1 e^{\delta x} + B'_1 e^{-\delta x} \qquad \delta = \sqrt{-\frac{2 m E}{\hslash^2}}$$
	
	We see that as $x \rightarrow -\infty$, the Term $B'_1$, as well as $B_3$ approaches $\infty$. Since the wave function cannot approach $\infty$, $B'_1 = B_3 = 0$ is a condition.
	%$$\psi = \begin{case}
	%	\psi_1 = B_1 e^{\delta x} & x < -L \\
	%	\psi_2 = A_2 e^{i k x} + A'_2 e^{-i k x} & -L < x < L \\
	%	\psi_3 = B'_3 e^{-\delta x} & L < x	
	%\end{case}$$
	  
\end{enumerate}

\begin{donotbrake}
	
\textbf{Boundary conditions:} We require, that the wave function is continuous, as well as it's spacial derivative. Therefore, we have:

$$\psi_1 (-L) =\psi_2(-L) \qquad \psi_2 (L) = \psi_3(L)$$
$$\frac{\partial}{\partial x}\psi_1 (-L) = \frac{\partial}{\partial x}\psi_2(-L) \qquad \frac{\partial}{\partial x}\psi_2 (L) =\frac{\partial}{\partial x}\psi_3(L)$$

\end{donotbrake}

\begin{mtabular}{C{0.45\columnwidth}|C{0.45\columnwidth}}
	\textbf{Even solutions}: only even (cosine) components &
	\textbf{Odd solutions}: only odd (sine) components \\
	$\fmm \abs{\cos \left(k L\right)} =\frac{k}{k_o}, \quad \tan (k L) > 0$ &
	$\fmm \abs{\sin \left(k L\right)} =\frac{k}{k_o}, \quad \tan (k L) > 0$ \\
	$\fmm k_0 = \sqrt{\frac{2 m V_0}{\hslash^2}}$ & 
	$\fmm k_0 = \sqrt{\frac{2 m V_0}{\hslash^2}}$ \\
	
	\begin{tikzpicture}
	\begin{axis}[xlabel=$k$,ylabel=$\abs{\cos(kL)}$, axis lines=middle, xmin = 0, xmax = 6, ymin = 0, ymax = 1.5, width = 0.5\columnwidth, height = 4cm,
	xtick ={0.8,5},
	xticklabels ={$k_0$, $k'_0$},
	ytick = {1}
	]
	\addplot [thick, samples=400, domain=0:5.8, color=cRed] {abs(cos(180*x/pi))};
	\draw [dashed] (axis cs:0,1) -- (axis cs:5.8,1) (axis cs:0.8,0) -- (axis cs:0.8,1) -- (axis cs:0,0) (axis cs:5,0) -- (axis cs:5,1) -- (axis cs:0,0);
	\draw [fill=black] (axis cs:0.64,0.8) circle (0.7mm);
	\draw [fill=black] (axis cs:1.31,0.26) circle (0.7mm);
	\draw [fill=black] (axis cs:3.84,0.77) circle (0.7mm);
	\end{axis}
	\end{tikzpicture} &
	
	\begin{tikzpicture}
	\begin{axis}[xlabel=$k$,ylabel=$\abs{\sin(kL)}$, axis lines=middle, xmin = 0, xmax = 6, ymin = 0, ymax = 1.5, width = 0.5\columnwidth, height = 4cm,
	xtick ={0.8,4},
	xticklabels ={$k_0$, $k'_0$},
	ytick = {1}
	]
	\addplot [thick, samples=400, domain=0:5.8, color=cRed] {abs(sin(180*x/pi))};
	\draw [dashed] (axis cs:0,1) -- (axis cs:5.8,1) (axis cs:0.8,0) -- (axis cs:0.8,1) -- (axis cs:0,0) (axis cs:4,0) -- (axis cs:4,1) -- (axis cs:0,0);
	\draw [fill=black] (axis cs:2.47,0.62) circle (0.7mm);
	\end{axis}
	\end{tikzpicture} \\
	
\end{mtabular}	

\subsection{Example: 1D potential step function}

\begin{tabular}{cc}
	\begin{tabular}{l}
		\begin{tikzpicture}
		\draw [->] (0,0.5) -- (4,0.5) node[right] {$x$};
		\draw [->] (2,0) -- (2,2.5) node[above] {$V(x)$};
		\draw [thick] (0,0.5) -- (2,0.5) -- (2,2) -- (4,2);
		\draw (1,0.25) node {1} (3,0.25) node[right] {2};
		\draw (2,2) node[above left] {$V_0$};
		\draw [thick, cRed] plot[domain=0:6*pi, samples=100]  (2-\x/pi/3,{-0.25*sin(\x r) + 1.6});
		\draw [thick, cRed, <-] (2,1.6) -- (1.95,1.52);
		\draw [thick, cGreen] plot[domain=0:6*pi, samples=100]  (2+\x/pi/3,{0.15*sin(\x*1.5 r) + 1.6});
		\draw [thick, cGreen, ->] (4,1.6) -- (4.05,1.53);
		\draw [thick, cBlue] plot[domain=0:6*pi, samples=100]  (2-\x/pi/3,{0.1*sin(\x r) + 0.9});
		\draw [thick, cBlue, <-] (-0.1,1) -- (-0.05,0.94); 
		\draw (0.85,1.5) node {\color{cRed} $A$} (1.15,0.8) node {\color{cBlue} $B$} (3,1.3) node {\color{cGreen} $C$};
		\end{tikzpicture}
	\end{tabular} &
	\begin{tabular}{p{0.5\columnwidth}}
		An incoming plane wave from the left hits a potential step at $x = 0$. 
		In region 1, two waves are added together, one is traveling to the right and one to the left.
		If $E > V_0$, the wave is transmitted to region 2. 
		if $E < V_0$, the wave decays exponentially in region 2.
	\end{tabular}
\end{tabular}

In \textbf{Region 1}, the general solution to the Schrödinger equation is:
$$\frac{-\hslash^2}{2m} \frac{\partial^2}{\partial x^2} \psi_1(x) = E \psi_1(x), \quad \psi_1(x) = A e^{i k_1 x} + B e^{-i k_1 x}, \quad k = \sqrt{\frac{2 m E}{\hslash^2}}$$

In \textbf{Region 2}, there are two cases, which can apply:
\begin{enumerate}
	\item $\mathbf{E > V_0}$: \textbf{Transmission}
	$$-\frac{\hslash^2}{2m} \frac{\partial^2}{\partial x^2} \psi_2 = (E - V_0) \psi_2(x) \qquad \psi_2 = C e^{i k_2 x}, \qquad k_2 = \sqrt{\frac{2 m (E - V_0)}{\hslash^2}}$$
	\item $\mathbf{E < V_0}$: \textbf{Complete reflection}
	$$-\frac{\hslash^2}{2m} \frac{\partial^2}{\partial x^2} \psi_2 = (E - V_0) \psi_2(x) \qquad \psi_2 = C e^{\delta_2 x}, \qquad \delta_2 = \sqrt{\frac{2 m (V_0 - 2)}{\hslash^2}}$$
\end{enumerate}

Applying the \textbf{initial conditions}, which require the wave function and it's derivative to be continuous at $x = 0$, we get the following expression for $A$, $B$, $C$:

$$\psi_1(x=0) = \psi_2(x=0) \qquad \frac{\partial }{\partial x} \psi_1(x=0) = \frac{\partial}{\partial x} \psi_2(x=0)$$

\begin{center}
\begin{mtabular}{c|c}
	$\mathbf{E > V_0}$ & $\mathbf{E < V_0}$ \\
	$A + B = C$ & $A + B = C$ \\
	$k_1 (A - B) = k_2 C$ & $A = B$ \\
\end{mtabular}
\end{center}

The \textbf{probability density function} $\abs{\psi(x,t)}^2 = \abs{\varphi(x)}^2 = \varphi \cdot \varphi^\ast$ can then be computed and sketched:

\begin{center}
\begin{mtabular}{c|c}
	$\mathbf{E > V_0}$ & $\mathbf{E < V_0}$ \\
	$\abs{\psi_1}^2 = A^2 + B^2  + 2AB \cos (2 k_1 x)$ & $\abs{\psi_1}^2 = 2 A^2 \cdot \left( 1 - \sin(2 k_1 x) \right)$ \\
	$\abs{\psi_2}^2 = C^2$ & $\fmm \abs{\psi_2}^2 = C^2 \cdot e^{-2 \delta x}$ \\
	\begin{tikzpicture}
		\begin{axis} [xlabel = $x$, ylabel=$\abs{\varphi}^2$, axis lines=middle, xmin = -2, xmax = 1, ymin = 0, ymax = 2, width=0.5\columnwidth, height=3cm, xtick = {0}, ytick = {0}]
			\addplot[thick, cRed, domain=-2:0, samples=99] {0.8 + 0.3*cos(800*x)};
			\addplot[thick, cRed, domain=0:2] {1.1};
		\end{axis}
	\end{tikzpicture} &
	\begin{tikzpicture}
	\begin{axis} [xlabel = $x$, ylabel=$\abs{\varphi}^2$, axis lines=middle, xmin = -2, xmax = 1, ymin = 0, ymax = 2, width=0.5\columnwidth, height=3cm, xtick = {0}, ytick = {0}]
	\addplot[thick, cRed, domain=-2:0, samples=99] {0.6 + 0.6*cos(800*x + 50)};
	\addplot[thick, cRed, domain=0:2] {e^(-7*x)};
	\end{axis}
	\end{tikzpicture}
\end{mtabular}
\end{center}

To find the \textbf{transmission coefficient} $T$ and the \textbf{reflection coefficient} $R$, we normalize $A = 1$. Then, we can define $B = \sqrt{R}$ and $C =\sqrt{T}$. Then, we can solve for $R$ and $T$:

$$T = \frac{4 k_1 k_2}{(k_1 + k_2)^2} \qquad R = \left(\frac{k_1 - k_2}{k_1 + k_2}\right)^2$$

If $E < V_0$, nothing is transmitted and therefore $T = 0$ and $R = 1$. 

\subsubsection{Example: 1D finite potential barrier}

\begin{tabular}{cc}
	\begin{tabular}{l}
		\begin{tikzpicture}
		\draw [->] (0,0.5) -- (4,0.5) node[right] {$x$};
		\draw [->] (2,0) -- (2,2.5) node[above] {$V(x)$};
		\draw [thick, cRed] (0,0.5) -- (1.5,0.5) -- (1.5,2) -- (2.5,2) -- (2.5,0.5) -- (4,0.5);
		\draw (1,1.25) node {1} (2,1.25) node[right] {2} (3,1.25) node[right] {3};
		\draw (2,2) node[above left] {$V_0$};
		\draw (1.5,0.25) node {$-\ell/2$} (2.5,0.25) node{$\ell/2$};
		%\draw [thick, cRed] plot[domain=0:6*pi, samples=100]  (2-\x/pi/3,{-0.25*sin(\x r) + 1.6});
		%\draw [thick, cRed, <-] (2,1.6) -- (1.95,1.52);
		%\draw [thick, cGreen] plot[domain=0:6*pi, samples=100]  (2+\x/pi/3,{0.15*sin(\x*1.5 r) + 1.6});
		%\draw [thick, cGreen, ->] (4,1.6) -- (4.05,1.53);
		%\draw [thick, cBlue] plot[domain=0:6*pi, samples=100]  (2-\x/pi/3,{0.1*sin(\x r) + 0.9});
		%\draw [thick, cBlue, <-] (-0.1,1) -- (-0.05,0.94); 
		%\draw (0.85,1.5) node {\color{cRed} $A$} (1.15,0.8) node {\color{cBlue} $B$} (3,1.3) node {\color{cGreen} $C$};
		\end{tikzpicture}
	\end{tabular} &
	\begin{tabular}{p{0.5\columnwidth}}
		An incoming plane wave from the left hits a potential barrier with length $l$. 
		The Transmission coefficient tells, how much of the wave can continue at the other side of the barrier (quantum tunneling). 
	\end{tabular}
\end{tabular}

In \textbf{Region 1 and 3}, the general expression for the wave equation is the following:
$$\psi_j(x) = A_j e^{i k_j x} + A'_j e^{-i k_j x}, \qquad k_j = \sqrt{\frac{2mE}{\hslash^2}}, \quad j \in \left\{ 1, 3 \right\}$$
In \textbf{Region 2}, the expression is depending on $V_0$. There are two cases:
\begin{enumerate}
	\item $\mathbf{E < V_0}$: $\fmm \varphi_2 = B_2 e^{\delta_2 x} + B'_2 e^{- \delta_2 x}, \qquad \delta_2 = \sqrt{\frac{2m(V_0 - E)}{\hslash^2}}$
	\item $\mathbf{E > V_0}$: $\fmm \varphi_2 = A_2 e^{i k_2 x} + A'_2 e^{-i k_2 x}, \qquad k_2 = \sqrt{\frac{2m(E - V_0)}{\hslash^2}}$
\end{enumerate}
Apply \textbf{boundary conditions} at $x = -\ell/2$ and $x = \ell/2$ in order to determine all constants. If the wave is only traveling from left to right, then $A'_3 = 0$.

$$\psi_1(-\ell/2) = \psi_2(-\ell/2), \quad \psi_2(\ell/2) = \psi_3(\ell/2)$$
$$\frac{\partial}{\partial x}\psi_1(-\ell/2) = \frac{\partial}{\partial x}\psi_2(-\ell/2), \quad \frac{\partial}{\partial x}\psi_2(\ell/2) = \frac{\partial}{\partial x}\psi_3(\ell/2)$$

Then, the \textbf{transmission coefficient} $T$ and the \textbf{reflection coefficient} $R$ can be calculated as following:

$$R = \left(\frac{A_1}{A'_1}\right)^2, \qquad T = \left(\frac{A_3}{A_1}\right)^2$$

\begin{center}
\begin{mtabular}{c|c}
	$\mathbf{E < V_0}$ & $\mathbf{E > V_0}$ \\
	$\fmm T = \frac{4E (V_0 - E)}{4 E (V_0 - E) + V_0^2 \sinh^2 ( \delta_2 \ell )}$ &
	$\fmm T = \frac{4E (V_0 - E)}{4 E (V_0 - E) + V_0^2 \sin^2 ( k_2 \ell )}$ \\
\end{mtabular}
\end{center}

If $\mathbf{E > V_0}$, the transmission coefficient has a maximum. If $k_2 \ell = n \pi \, \Rightarrow \, T = 1$ (\textbf{resonance}). The minimum of $T$u is at: $k_2 \ell = \pi/2 + n\pi$.

If $\delta_2 \ell \gg 1$, the transmission coefficient is proportional to: $T \propto e^{-2 \delta_2 \ell}$

\end{twocolumn}

\section{Wave Function Space (Hilbert Space)}

\subsection{Inner Product}

The inner product $\braket{\psi_1 | \psi_2}$ is defined like the scalar product for vectors. If the inner product $\braket{\psi_1 | \psi_2} = 0$ , $\psi_1$, $\psi_2$ are \textbf{orthogonal}.

$$\braket{\psi_1 | \psi_2} = \infint \psi_1^\ast(\bm{x},t) \psi_2(\bm{x},t) d^3 \bm{x}$$
$$\braket{\psi | \psi} = \infint \psi^\ast(\bm{x},t) \psi(\bm{x},t) d^3 \bm{x} = \infint \abs{\psi(\bm{x},t)}^2 d^3 \bm{x} = 1$$


\subsection{Fourier Transform}

$$\psi(x) = \frac{1}{\sqrt{2 \pi \hslash}} \infint e^{\frac{i p x}{\hslash}} \varphi(p) dp, \quad \varphi(p) = \frac{1}{\sqrt{2 \pi \hslash}} \infint e^{\frac{i p x}{\hslash}} \psi(x) dx$$
$$\psi(\bm{\vec{x}}) = \frac{1}{(2 \pi \hslash)^{3/2}} \infint e^{\frac{i \bm{\vec{p}} \bm{\vec{x}}}{\hslash}} \varphi(\bm{\vec{p}}) d\bm{\vec{p}}, \quad \varphi(\bm{\vec{p}}) = \frac{1}{(2 \pi \hslash)^{3/2}} \infint e^{\frac{i \bm{\vec{p}} \bm{\vec{x}}}{\hslash}} \psi(\bm{\vec{x}}) d\bm{\vec{x}}$$

$$\infint \psi_1^\ast(x) \cdot \psi_2(x) \cdot dx = \infint \varphi^\ast_1(p) \cdot \varphi_2(p) \cdot dp$$

\section{Observable Measurements, Time-dependence}

Doing a measurement in quantum mechanics (observable) can be interpreted as applying an operator $\hat{A}$ on the wave function $\psi(\bm{x},t)$. For example, tu o compute the expected position $\langle \bm{x} \rangle_\psi$, we apply the operator $\hat{\bm{x}} = \bm{x}$ to average the wave function:
$$\langle \bm{x} \rangle_\Psi = \iiint \Psi^\ast(\bm{x},t) \cdot \bm{x} \cdot \Psi(\bm{x},t) d^3 \bm{x} = \iiint \bm{x} \cdot \abs{\Psi(\bm{x},t)}^2 d^3\bm{x}$$

\begin{center}
	\begin{tabular}{lll}
		Name & Operator \\ \toprule
		Position & $\hat{\bm{x}} = \left[ \bm{x} \right]$ \\
		Momentum & $\hat{\bm{p}} = \left[-i \hslash \bm{\nabla}\right]$ & $\bm{\nabla} = \Vector{\pdif{}{x} & \pdif{}{y} & \pdif{}{z}}^T$ \\
		Hamiltonian & $\hat{H} = \left[ -\frac{\hslash^2}{2m} \nabla^2 + V(\bm{x}) \right] $ & $\nabla^2 = \pdiff{}{x} + \pdiff{}{y} + \pdiff{}{z}$
	\end{tabular}
\end{center}

\subsection{Canonical commutation relation}

The commutators is a way of describing the effect of the order, in which multiple operators are applied.
$$\left[\hat{A},\hat{B}\right] = \hat{A} \hat{B} - \hat{B} \hat{A}, \quad \left[\hat A,\hat B\right] = -\left[\hat B,\hat A\right], \quad \left[\hat A,\hat A\right] = 0$$
$$\left[\hat A,(\hat B+\hat C)\right] = \left[\hat A,\hat B\right] + \left[\hat A,\hat C\right]$$
$$\left[\hat p_x,\hat p_y\right] = 0, \ \left[\hat x,\hat p_x\right] = i\hslash, \ \left[\hat z,\hat p_x\right] = \left[\hat z,\hat p_y\right] = 0$$

\subsection{Eigenstates and Eigenvalues}

An Observable has an Operator $\hat{A}$. a state $u_n(x)$ is called an eigenstate the operator applied on the wave function acts like a scalar multiplication to it. Then, the measurement of the general state $\psi(x)$ is a superposition of all the eigenstates.
$$\hat{A} u_n(x) = a_n u_n(x), \quad \infint u^\ast_n(x) \hat{A} u_n(x) dx = a_n \quad \hat{A} \psi(x) = \sum_n c_n u_n(x)$$


\subsection{Harmonic Oscillator}

A Quantum mechanical harmonic oscillator can be interpreted as the solution to the Schrödinger equation:
$$\left[ \frac{-\hslash^2}{2m} \pdiff{}{x}  + V(x) \right] \psi(x) = E \psi(x), \quad V(x) = \frac{1}{2} k x^2 = \frac{m \omega^2}{2} x^2$$

To simplify the equation, we define a new length scale and energy:
$$a = \sqrt{\frac{\hslash}{m \omega}}, \quad \tilde{x} = \frac{x}{a}, \quad \tilde{E} = \frac{E}{\hslash \omega} \, \Rightarrow \, \frac{1}{2} \left[-\pdiff{}{\tilde{x}} + \tilde{x}^2\right] \varphi(\tilde{x}) = \tilde{E} \varphi(\tilde{x})$$

Then, the solutions to the equation is:
$$E_n = \left(n + \frac{1}{2}\right) \hslash \omega, \quad \psi(\tilde{x}) = c_n H_n(\tilde{x}) e^{-
\tilde{x}/2}, \quad H_n(\tilde{x}) = (-1)^n e^{\tilde{x}^2} \cdot \frac{\partial^n}{\partial \tilde{x}^n} e^{-\tilde{x}^2}$$
$$H_0(\tilde{x}) = 1, \quad H_1(\tilde{x}) = 2 \tilde{x}, \quad H_2(\tilde{x}) = 4 \tilde{x}^2 - 2, \quad H_3(\tilde{x}) =8 \tilde{x}^3 - 12\tilde{x}$$
$$\Psi_n(x) = \frac{1}{\sqrt[4]{\pi} \sqrt{2^n n! a}} \cdot H_n\left(\frac{x}{a}\right) e^{-\frac{x^2}{2a^2}}$$

\begin{center}
	\begin{tikzpicture}
	\begin{axis}[xlabel=$\frac{x}{a}$,ylabel=$E$, axis lines=middle, xmin = -4, xmax = 4, ymin = 0, ymax = 10, width=0.7\columnwidth, height=5.5cm, xtick = \empty, ytick = {1, 3, 5, 7}, yticklabels = {$\frac{\hslash \omega}{2}$}]
		\addplot [thick, color=cRed, samples=50] {x^2*3/2};
		\addplot [color=cGreen, samples=50, domain=-3.5:3.5] {exp(-x^2) + 1};
		\addplot [color=cGreen, samples=50, domain=-3.5:3.5] {exp(-x^2)*x*3/2 + 3};
		\addplot [color=cGreen, samples=50, domain=-3.5:3.5] {exp(-x^2)*(2*x^2-1) + 5};
		\addplot [color=cGreen, samples=50, domain=-3.5:3.5] {exp(-x^2)*(2*x^3-3*x)*2/3 + 7};
	\end{axis}
	\end{tikzpicture}
\end{center}

\subsection{The coupled quantum well}

\begin{tabular}{cc}
	\begin{tabular}{c}
		\begin{tikzpicture}
			\draw [->] (-2,0) -- (2,0) node[right] {$x$};
			\draw [->] (0,0) -- (0,2) node[right] {$V(x)$};
			\draw [thick, cRed, <->] (-1.5,2) -- (-1.5,0) -- (-0.5,0) -- (-0.5,1) -- (0.5,1) -- (0.5,0) -- (1.5,0) -- (1.5,2);
			\draw (-1,1.5) node{\rom{1}} (0,1.5) node[right] {\rom{2}} (1,1.5) node{\rom{3}};
			\draw (-1,0.1) node[above] {$-b$} -- (-1,-0.1);
			\draw ( 1,0.1) node[above] {$b$}  -- ( 1,-0.1);
			\draw [|-|] (-1.5,-0.4) -- node[pos=0.5, fill=white] {$a$} (-0.5,-0.4);
			\draw [|-|] ( 1.5,-0.4) -- node[pos=0.5, fill=white] {$a$} ( 0.5,-0.4);
			\draw       (-0.5,-0.4) -- node[pos=0.5, fill=white] {$\Delta$} ( 0.5,-0.4);
		\end{tikzpicture}
	\end{tabular} &
	\begin{tabular}{p{0.5\columnwidth}}
		This is the simplified potential of an ammonia molecule $\chem{NH_3}$. The wave function outside the well ($\abs{x} > b + \frac{a}{2}$) is zero. There exists a symmetric, as well as an antisymmetric solution. We consider the case: $\bm{E < V_0}$
	\end{tabular}
\end{tabular}

%$$\psi_{\rom{1}} = \pm \lambda \cdot \sin\left(k \left(b - \frac{a}{2} + x\right)\right) \qquad \psi_{\rom{3}} = \pm \lambda \cdot \sin\left(k \left(b - \frac{a}{2} + x\right)\right)$$
$$\psi_{\rom{2}} = \begin{case} \mu \cosh (\delta x)  & \text{symmetric} \\ \mu \sinh (\delta x) & \text{antisymmetric}\end{case} \quad k = \sqrt{\frac{2mE}{\hslash^2}}, \quad \delta = \sqrt{\frac{2 m (V_0 - E)}{\hslash^2}}$$

\begin{tabular}{cc}
	\begin{tabular}{c}
		\begin{tikzpicture}
			\begin{axis}[xlabel=$k a$,ylabel=$\tan(k a)$, xmin=0, xmax=5*pi/2, ymin=-6, ymax=6, width=0.45\columnwidth, height=0.4\columnwidth, axis lines=middle, ticks=none, every axis x label/.style={at={(ticklabel* cs:1.05)}, anchor=west}, every axis y label/.style={at={(ticklabel* cs:1.05)}, anchor=south}]
				\addplot[thick, color=cRed, samples=20, domain=0:(pi/2-0.1)] {tan(x*180/pi)};
				\addplot[thick, color=cRed, samples=40, domain=(1*pi/2+0.1):(3*pi/2-0.1)] {tan(x*180/pi)};
				\addplot[thick, color=cRed, samples=40, domain=(3*pi/2+0.1):(5*pi/2-0.1)] {tan(x*180/pi)};
				\draw [cRed, dashed] (axis cs:pi/2,6) -- (axis cs:pi/2,-6) (axis cs:3*pi/2,6) -- (axis cs:3*pi/2,-6);
				\addplot[thick, color=cBlue, samples=2, domain=0:5*pi/2] {-0.3*x};
				\addplot[thick, color=cGreen, samples=2, domain=0:5*pi/2] {-0.4*x};
			\end{axis}
		\end{tikzpicture}
	\end{tabular} &
	\begin{mtabular}{c}
		$\fmm \color{cGreen} \text{symmetric: }\varepsilon_s = \frac{1 + e^{-\delta \Delta}}{\delta a}$ \\
		$\fmm \color{cBlue}\text{antisymmetric: } \varepsilon_a = \frac{1 - e^{-\delta \Delta}}{\delta a}$ \\
		$\fmm \tan(k a) = -k a \varepsilon = -k a \frac{1 \pm e^{-\delta \Delta}}{\delta a}$
	\end{mtabular}
\end{tabular}

Now, we can create a superposition of both the symmetric and the antisymmetric case:

$$\psi_{s_\rom{1}} = + \lambda \sin\left( k \left( b - \frac{a}{2} + x \right) \right), \quad \psi_{s_\rom{3}} = + \lambda \sin\left( k \left( b - \frac{a}{2} + x \right) \right)$$
$$\psi_{a_\rom{1}} = - \lambda \sin\left( k \left( b - \frac{a}{2} + x \right) \right), \quad \psi_{a_\rom{3}} = + \lambda \sin\left( k \left( b - \frac{a}{2} + x \right) \right)$$
$$\Psi_L = \frac{1}{\sqrt{2}} (\Psi_s - \Psi_a), \qquad \Psi_R = \frac{1}{\sqrt{2}} (\Psi_s + \Psi_a)$$
$$\Psi_L(x,t) = \frac{1}{\sqrt{2}} e^{-i \omega_s t} \left(\psi_s(x) - e^{-i (\omega_a - \omega_s) t} \psi_a(x)\right)$$
$$\omega_a = \frac{E_a}{\hslash}, \quad \omega_s = \frac{E_a}{\hslash}, \quad E_a - E_s = \frac{\hslash^2 \pi^2}{2 m \delta a^2} \cdot 8 e^{-\delta \Delta}$$

From the formula describing the wave equation, we can see that at $t_0$, the particle can only be found in region \rom{1}, and after some time $t_{1/2}$, the particle can only be found in region \rom{3}. The particle has tunneled from one side to the other. Now, we can define a period $T = \frac{2 \pi \hslash}{E_a - E_s}$

\section{Schrödinger Equation in 3D}

\begin{tabular}{cc}
	\begin{tabular}{c}
		\begin{tikzpicture}[scale=0.8, x={(350:1cm)},y={(40:0.8cm)},z={(90:1cm)}]
			\draw [->] (0,0,0) -- (3.5,0,0) node[right] {$x_1$};
			\draw [->] (0,0,0) -- (0,4,0) node[right] {$x_2$};
			\draw [->] (0,0,0) -- (0,0,2.5) node[right] {$x_3$}; 
			\draw [thick,cRed] (0,0,0) -- node[black, pos=0.5, below] {$L_1$} (2,0,0) -- node[black, pos=0.5, below right] {$L_2$} (2,1.8,0) -- (0,1.8,0) -- (0,0,0) -- node[black, pos=0.5, left] {$L_3$} (0,0,1) -- (2,0,1) -- (2,1.8,1) -- (0,1.8,1) -- (0,0,1) (2,0,1) -- (2,0,0) (2,1.8,1) -- (2,1.8,0) (0,1.8,1) -- (0,1.8,0);
		\end{tikzpicture}
	\end{tabular} &
	\begin{mtabular}{c}
		$V(x_i) = \begin{case} 0 & \cif 0 < x_i < L_i \\ \infty & \celse	\end{case}$ \\
		$V(x,y,z) = V(x_1) + V(x_2) + V(x_3)$ \\
		$\psi(x_1,x_2,x_3) = \psi_1(x_1) \cdot \psi_2(x_2) \cdot \psi_3(x_3)$
	\end{mtabular}
\end{tabular}

$$-\frac{\hslash^2}{2m} \left[ \frac{\psi_1''(x_1)}{\psi_1(x_1)} + \frac{\psi_2''(x_2)}{\psi_2(x_2)} + \frac{\psi_3''(x_3)}{\psi_3(x_3)} \right] + V(x_1) + V(x_2) + V(x_3) = E$$

This equation can be separated into three smaller equations for every spacial dimension $x_i$

$$-\frac{\hslash^2}{2m} \pdiff{}{x_i}\psi_i(x_i) +V(x_i) \psi_i(x_i) = E_i \psi_i(x_i) $$
$$E_i^{(n_i)} = n_i^2 \frac{\hslash^2 \pi^2}{2m L_i^2}, \qquad \psi_i^{(n_1)} = A \cdot \sin \left( \frac{\pi n_i x}{L_i} \right)$$

After normalizing, the wave function can be written as:

$$\psi(x_1,x_2,x_3) = \sqrt{\frac{8}{L_1 L_2 L_3}} \sin \left(\frac{\pi n_1 x_1}{L_1}\right)\sin \left(\frac{\pi n_2 x_2}{L_2}\right) \sin \left(\frac{\pi n_3 x_3}{L_3}\right)$$

When $L_1 = L_2 =L_3$, there sometimes exists multiple states (\textbf{degeneracies}) for the same energy $E = E_1 + E_2 + E_3$. 
Now, we can generate new solutions to the wave function via superposition of those states. 
In general, degeneracies arise from symmetries (obvious or hidden).

\subsection{Schrödinger Equation in spherical coordinates}

\begin{center}
	\begin{tabular}{cc}
		\begin{tabular}{c}
			\begin{tikzpicture}
				\draw [->] (0,0) -- (-1,-0.8) node[right] {$x$};
				\draw [->] (0,0) -- (2,0) node[right] {$y$};
				\draw [->] (0,0) -- (0,2) node[right] {$z$};
				\draw [fill=black] (30:1.5) circle (0.1cm);
				\draw (0,0) -- node[pos=0.5, below] {$r$} (30:1.5);
				\draw [dashed] (30:1.5) -- ++(0,-1.5) -- (0,0);
				\draw [xscale=0.8, xshift=0.22cm] (30:1) arc (30:103:1) node[pos=0.5, below left] {$\theta$};
				\draw [yscale=0.5, yshift=-0.4cm] (-30:0.8) arc (-30:-138:0.8) node[pos=0.5, above] {$\varphi$};
			\end{tikzpicture}
		\end{tabular} &
		\begin{mtabular}{c}
			$x = r  \sin \theta \cos \varphi$ \\
			$y = r \sin \theta \sin \varphi$ \\
			$z = r \cos \theta$ \\
		\end{mtabular}
	\end{tabular}
\end{center}

\begin{comment}
To use the Schrödinger equation, we must define the Laplacian operator $\nabla^2$:
$$\nabla^2 = \frac{1}{r^2} \pdif{}{r} \left(r^2 \pdif{}{r}\right) + \frac{1}{r^2 \sin\theta} \pdif{}{\theta} \left(\sin\theta \pdif{}{\theta}\right) + \frac{1}{r^2 \sin^2\theta} \pdiff{}{\varphi}$$

Now, we insert this into the Schrödinger equation and try to separate the radial part $R(r)$ from the angular part $Y(\theta, \varphi)$. By introducing a separation constant $\ell(\ell+1)$, we get:

$$\psi(r,\theta, \varphi) = R(r) \cdot Y(\theta, \varphi)$$
{\small $$\frac1R \pdif{}{r} \left(r^2 \pdif{R}{r}\right) - \frac{2mr^2}{\hslash^2} (V-E) = - \left( \frac{1}{Y \sin \theta} \pdif{}{\theta} \left( \sin \theta \pdif{Y}{\theta} \right) + \frac{1}{\sin^2 \theta} \pdiff{Y}{\varphi} \right) =\ell(\ell+1)$$}

The angular equation can be rewritten, in order to separate $Y(\theta, \varphi)$ into $\Theta(\theta)\Phi(\varphi)$. With this separation, we get for the angular part:
$$\Phi(2\pi) = \Phi(0) \; \rightarrow \; m \in \mathbb{Z}, \quad \abs{m} \leq \ell$$
\end{comment}

$$\psi_{n \ell m}(r,\theta,\varphi) = R_{n \ell}(r) \cdot Y_\ell^m (\theta, \varphi) = R_{n\ell}(r) \cdot P_\ell^m(\cos \theta) e^{i m \varphi}$$

The angular part $Y_\ell^m(\theta, \varphi)$ can be written as:
$$P_\ell^m(x) = (i-x^2)^{\frac{\abs{m}}{2}} \frac{d^{\abs{m}}}{dx^{\abs{m}}} P_\ell(x) \qquad P_\ell(x) = \frac{1}{2^\ell \cdot \ell!} \frac{\partial^\ell}{dx^\ell} (x^2-1)^\ell$$

The solution to $Y$ will be a \textbf{spherical harmonic}.Finally, we must apply the normalization
$$\int_0^\infty \abs{R(r)}^2 r^2 dr = 1, \qquad \int_{\theta=0}^{\pi} \int_{\varphi=-\pi}^{\pi} \abs{Y_\ell^m(\theta,\varphi)}^2 \sin \theta d\varphi d\theta = 1$$

These solutions are the same as \textbf{spherical harmonics}. 
They form an \textbf{orthogonal basis}, meaning that every well-behaved function $f(\theta, \varphi)$ can be expressed as a superposition of those harmonics.

%$$f(\theta, \varphi) = \sum_{\ell = 0}^{\infty} \sum_{m = -\ell}^{\ell} C_\ell^m Y_\ell^m(\theta, \varphi)$$
\subsubsection{Hydrogen Atom}

The radial part $R_{n\ell}$ of the hydrogen atom with potential $V(r) = \frac{-e^2}{4\pi\epsilon_0 r}$ can be written as:
$$R_{n\ell}(r) = \frac{1}{r} \rho^{\ell+1} e^{-\rho} v(\rho), \quad \rho = \frac{r}{n a_0}, \quad a_0 = \frac{4\pi \epsilon_0 \hslash^2}{m e^2} \approx  5.29 \cdot 10^{-11} \si{\meter}$$
$$\psi_{n\ell m}(r,\theta,\varphi) = R_{n\ell}(r) Y_\ell^m(\theta,\varphi) \qquad  j_{max} = (n-\ell-1) \geq 0 \qquad \abs{m} \leq \ell$$
$$E = -\frac{E_{Ry}}{n^2} \approx -\frac{13.6}{n^2} \si{\electronvolt}$$

$v(\rho)$ is a polynomial of degree $j_{max}$ with coefficients: $C_{g+1} = \frac{2(g+l+1-n)}{(g+1)(g+2l+2)} C_g$. 
For state $n$, there are $d(n) = n^2$ different solutions (\textbf{degeneracies}). 
The \textbf{effective radius} is $n a_0$.
The \textbf{probability} of of finding an electron between $r$ and $r+dr$ is:
$$p(r) dr = r^2 \abs{R_{n\ell}(r)}^2 dr$$

\subsubsection{Quantum Numbers}

$n$ is the main quantum number, $\ell$ is the orbital quantum number and $m$ is the magnetic quantum number (projection of angular momentum).
Chemists give the different $\ell$'s different names. 
\begin{itemize}
	\item $\ell = 0$: the orbital is called an s-state ($\max p(r) dr$ is at $r = 0$). 
	\item $\ell = 1$: the orbital is called an p-state ($p(r=0) dr = 0$).
	\item $\ell = 2$: the orbital is called an d-state.
\end{itemize}

\section{Angular Momentum and Spin}

$$ \begin{array}{lll}\hat L_x = \hat y \hat p_z - \hat z \hat p_y & \hat L_y = \hat z \hat p_x - \hat x \hat p_z & \hat L_z = \hat x \hat p_y - \hat y \hat p_x \\ \left[\hat L_x,\hat L_y\right] = i \hslash \hat L_z & \left[\hat L_y,\hat L_z\right] = i \hslash \hat L_x & \left[\hat L_z,\hat L_x\right] = i \hslash \hat L_y \\ \end{array} \hat{\bm{L}} = \det \left| \begin{array}{ccc} \bm{e_x} & \bm{e_y} & \bm{e_z} \\ \hat x & \hat y & \hat z \\ \hat p_x & \hat p_y & \hat p_z\end{array} \right| $$

\small {
$$\hat L_x = i \hslash \left( \sin \varphi \pdif{}{\theta} + \frac{\cos \theta \cos \varphi}{\sin \theta} \pdif{}{\varphi}  \right), \ \hat L_y = -i \hslash \left( \cos \varphi \pdif{}{\theta} - \frac{\cos \theta \sin \varphi}{\sin \theta} \pdif{}{\varphi} \right)$$
$$\hat L_z = -i \hslash \pdif{}{\varphi}, \quad \hat L^2 = -\hslash^2 \left[\frac{1}{\sin \theta} \pdif{}{\theta} \left( \sin \theta \pdif{}{\theta} \right) + \frac{1}{\sin^2 \theta} \pdiff{}{\varphi} \right]$$
}

Angular momentum operators do not commute. In order to get commutable operators, we introduce $\hat L^2 = \hat L_x^2 + \hat L_y^2 + \hat L_z^2$

$$\left[\hat L^2,\hat L_x\right] = 0, \qquad \left[\hat L^2,\hat L_y\right] = 0, \qquad \left[\hat L^2,\hat L_z\right] = 0$$

$$\hat L^2 Y_\ell^m (\theta,\varphi) = \hslash^2 \ell (\ell + 1) Y_\ell^m(\theta,\varphi), \qquad \hat L_z Y_\ell^m(\theta,\varphi) = \hslash m Y_\ell^m(\theta,\varphi)$$

\subsection{Ladder Operator}

If a ladder operators $\hat L_\pm = \hat L_x \pm i \hat L_y$ are used in the following way: Suppose, we have a wave function $\psi$, which is simultaneously an eigenfunction of $\hat L^2$ and $\hat L_z$. Then, $\hat L_\pm \psi$ is also an eigenfunction of $\hat L^2$ and $\hat L_z$ with the following eigenvalues:

$$\hat L^2 \psi = \lambda \psi, \quad \hat L_z \psi = \mu \psi \qquad \hat L^2 (\hat L_\pm \psi) = \lambda (\hat L_\pm \psi), \quad \hat L_z (\hat L_\pm \psi) = (\mu \pm \hslash) (\hat L_\pm \psi)$$ 
$$\hat L_+ Y_\ell^m = \hslash \sqrt{\ell (\ell + 1) - m (m + 1)} Y_\ell^{m+1} \qquad \hat L_- Y_\ell^m = \hslash \sqrt{\ell (\ell + 1) - m (m - 1)} Y_\ell^{m-1}$$

\subsection{Spin}

Idea: $\ell = \frac12 \ \rightarrow \ m = \pm \frac12$. Instead of using $\ell$, we use $s$ to describe the spin. The operators $\hat L_i$ are now called $\hat S_i$. We define the spin as $\ket{s,m_s}$:
$$\chi_+ = \Ket{\frac12,\frac12} = \ket{\uparrow}, \quad \chi_- = \Ket{\frac12,-\frac12} = \ket{\downarrow}, \quad \chi = a \chi_+ + \beta \chi_- = \begin{pmatrix}a\\b\end{pmatrix}, \ \abs{a}^2 + \abs{b}^2 = 1$$

Since we have now only two eigenstates $\spinup$ and $\spindown$, we can write:
$$\hat S^2 = \frac{3}{4} \hslash^2 \begin{pmatrix} 1&0\\0&1 \end{pmatrix} \quad \hat S_x = \frac{\hslash}{2} \begin{pmatrix} 0&1\\1&0 \end{pmatrix} \quad \hat S_y = \frac{\hslash}{2} \begin{pmatrix} 0&-i\\i&0 \end{pmatrix} \quad \hat S_z = \frac{\hslash}{2} \begin{pmatrix} 1&0\\0&-1 \end{pmatrix}$$

If we separate the spin in all three directions, we can write:
$$\ket{\uparrow_x} = \frac{1}{\sqrt{2}} \left(\ket{\uparrow_z} + \ket{\downarrow_z} \right), \qquad \ket{\downarrow_x} = \frac{1}{\sqrt{2}} \left(\ket{\uparrow_z} - \ket{\downarrow_z} \right)$$

Now, we define the ladder operators in the same way as for the angular momentum:
$$\hat S_+ = \begin{pmatrix} 0&\hslash\\0&0 \end{pmatrix} \quad \hat S_- = \begin{pmatrix} 0&0\\\hslash&0 \end{pmatrix} \quad \hat S_+ \spinup = \hat S_- \spindown = 0, \ \hat S_+ \spindown = \hslash \spinup, \ \hat S_- \spinup = \hslash \spindown$$

We can write states states in dirac notation as: $\ket{\ell,m}$. For spins, we get:
$$\ket{1,1} = \spinupup, \ \ket{1,-1} = \spindowndown, \ \ket{1,0} = \frac{1}{\sqrt2}(\spinupdown + \spindownup), \ \ket{0,0} = \frac{1}{\sqrt2} (\spinupdown - \spindownup) $$

\subsection{Identical Particles}
When we have two particles with wave function $\varphi_a(\vec x_1)$ and $\varphi_b(\vec x_2)$, we can combine them in two different ways:
\begin{itemize}
	\item \textbf{Fermion}: $\psi_- = \frac{1}{\sqrt2} \left( \varphi_a (\vec x_1) \varphi_b(\vec x_2) - \varphi_b(\vec x_1) \varphi_a (\vec x_2) \right)$: symmetric (electron)
	\item \textbf{Boson}: $\psi_+ = \frac{1}{\sqrt2} \left( \varphi_a (\vec x_1) \varphi_b(\vec x_2) + \varphi_b(\vec x_1) \varphi_a (\vec x_2) \right)$: antisymmetric (photon)
\end{itemize}
Pauli exclusion principle: \textbf{Two fermions cannot occupy the same identical state}

\subsubsection{Exchange Interactions}

We have two particles, which have normalized and orthogonal wave functions. We have three different interactions:
\begin{enumerate}
	\item The particles are \textbf{distinguishable}: $\psi = \psi_a(x_1) \psi_b(x_2)$
	$$\Braket{(x_1-x_2)^2}_\psi = \braket{x^2}_{\psi_a} + \braket{x^2}_{\psi_b} - 2 \braket{x}_{\psi_a} \braket{x}_{\psi_b}$$
	\item \textbf{Symmetric} wave function: $\psi_+ = \frac{1}{\sqrt2} (\ket{\psi_a} \ket{\psi_b} + \ket{\psi_b} \ket{\psi_a})$
	$$\Braket{(x_1-x_2)^2}_\psi = \braket{x^2}_{\psi_a} + \braket{x^2}_{\psi_b} - 2 \braket{x}_{\psi_a} \braket{x}_{\psi_b} - 2 \abs{\braket{\psi_a|x|\psi_b}}^2$$
	\item \textbf{Antisymmetric} wave function: $\psi_- = \frac{1}{\sqrt2} (\ket{\psi_a} \ket{\psi_b} - \ket{\psi_b} \ket{\psi_a})$
	$$\Braket{(x_1-x_2)^2}_\psi = \braket{x^2}_{\psi_a} + \braket{x^2}_{\psi_b} - 2 \braket{x}_{\psi_a} \braket{x}_{\psi_b}  + 2 \abs{\braket{\psi_a|x|\psi_b}}^2$$
\end{enumerate}

\subsection{Many Electrons: Atomic Shells}

To write the wave function of an electron, we use the notation indexed by: $\ket{n, \ell, m}$.
\begin{center}
	\begin{stabular}{cccc}
		shell $n$ & subshell $\ell$ & max $e^-$ in subshell & max $e^-$ in shell \\ \toprule
		K & 1s & 2 & 2 \\ \midrule
		L & 2s & 2 & $2+6=8$ \\ 
		& 2p & 6 & \\ \midrule
		M & 3s & 2 & $2+6+10=10$ \\
		& 3p & 6 & \\
		& 3d & 10 & \\
	\end{stabular}
\end{center}

Here, the number of degeneracies per shell is displayed. Remember, we have $n \geq 1$, $0 \leq \ell < n$, $-\ell \leq m \leq \ell$, and for every different state, the electron can have either spin up or spin down. So, the number of electrons in a sub shell is the number of degeneracies. 

\subsection{Term Symbol}

$$^{2S+1}L_j$$

$S$ is the total spin quantum number, $2S+1$ is the spin multiplicity, $J$ ($m$) is the total angular momentum quantum number and $L$ is the orbital quantum number in spectroscopic notation: $L=0 \rightarrow S$, $L=1 \rightarrow P$, $L=2 \rightarrow D$...

\section{Quantum Statistics and Solid States}

\subsection{Chemical Potential}

For the chemical potential $\mu$, the flux of particles (instead of energy) is important. 
$$\mu = \pdif{F}{N}, \qquad F = U - TS$$
where $F$ is the free energy, $U$ is the total energy, $T$ is the temperature, $S$ is the entropy and $N$ is the number of particles. In other words: \textbf{Chemical Potential is the energy exchanged when a particle is added or removed}.

\subsection{Fermi-Dirac \& Einstein Statistics}

What is the probability of a given state to be occupied? And if we have multiple particles, how many electrons are in a state?
$$\textbf{Fermions:} \ f_F(E, \mu, T) = \frac{1}{\exp \left( \frac{E-\mu}{kT} \right) + 1}, \quad \textbf{Bosons:} \ f_B(E, \mu, T) = \frac{1}{\exp \left( \frac{E-\mu}{kT} \right) - 1}$$  

As the energy $E \gg \mu$, both distributions are the same: the classical Bolzmann distribution. If the electrons are far away, we do not need to consider the interaction between single particles. 

\subsection{Fermi-Dirac statistics of a free electron gas}

The number of particles $N$ given an energy $E$ with temperature $T = 0$ is:
$$N(E) = \frac{L^3}{3\pi^2} \frac{(2mE)^\frac{3}{2}}{\hslash^2} = \frac{4}{3}\pi k^3 \cdot \frac18 \cdot \left(\frac{L}{\pi}\right)^3 \cdot 2$$

This equation is derived by multiplying the Volume of a sphere (in the positive $k$ sector) with radius $k$ with the volume of a single state and the number of possible spins (2). 

The Density of states $D(E)$ is given by:
$$D(E) = \pdif{N(E)}{E} \cdot \frac{1}{V} = \frac{1}{2\pi^2} \left(\frac{2m}{\hslash}\right)^{\frac{3}{2}} \sqrt{E}$$

\subsection{Bloch's Theorem}

In a periodic potential, the wave equation can be written as a periodic function $u_{nk}(x)$ multiplied by a complex exponential:
$$\psi_{nk}(x) = e^{ikx} u_{nk}(x)$$

We can apply this theorem the band formation to conclude the following. As soon as we have a crystal, the electrons can be in a continuous band of energies:
$$E = E_0 - 2A \cos(ka) = E(k)$$

where $A$ is the inter-atomic coupling. We can see that the band width is directly proportional to $A$. 


\section{Useful formulas}

$$\infint e^{-a x^2} dx = \sqrt{\frac{\pi}{a}} \qquad \int_0^\infty x e^{-a x^2} dc = \frac{1}{2a} \qquad \infint x^2 e^{-a x^2} = \frac{\sqrt{\pi}}{2a^{3/2}}$$
$$\int x^n e^{cx} = e^{cx} \sum_{i=0}^{n} (-1)^{n-i} \frac{n!}{i! c^{n-i+1}} x^i \qquad \int_0^\infty x^n e^{-cx} = \frac{n!}{c^{n+1}}$$
$$\text{Gaussian:} \quad G = A \cdot e^{\frac{-x^2}{2\sigma^2}}$$

\subsection{Trigonometry}
{\footnotesize
\begin{tabular}{r<{\hspace{-8pt}}c<{\hspace{-8pt}}lr<{\hspace{-8pt}}c<{\hspace{-8pt}}l}
	$\sin(2\alpha) $&$=$&$ 2 \sin \alpha \cos \alpha$ &
	$\cos(2\alpha) $&$=$&$ \cos^2 \alpha - \sin^2 \alpha$ \\
	$\sin(\alpha \pm \beta)$&$=$&$\sin(\alpha) \cos(b) \pm \cos(\alpha) \sin(b)$ &
	$\cos(\alpha \pm \beta)$&$=$&$\cos(\alpha) \cos(b) \mp \sin(\alpha) \sin(b)$ \\
	$\sin(\alpha)\pm \sin(\beta)$&$=$&$2 \sin \frac{\alpha \pm \beta}{2} \cos \frac{\alpha \mp \beta}{2}$ &
	$\cos(\alpha)+\cos(\beta)$&$=$&$2 \cos \frac{\alpha+\beta}{2} \cos\frac{\alpha-\beta}{2}$ \\
	$\cos(\alpha)-\cos(\beta)$&$=$&$-2 \sin \frac{\alpha+\beta}{2} \sin\frac{\alpha-\beta}{2}$ &
	$\sin(\alpha)\sin(\beta)$&$=$&$\frac{1}{2}(\cos(\alpha-\beta)-\cos(\alpha+\beta))$ \\
	$\cos(\alpha)\cos(\beta)$&$=$&$\frac{1}{2}(\cos(\alpha-\beta)+\cos(\alpha+\beta))$ &
	$\sin(\alpha)\cos(\beta)$&$=$&$\frac{1}{2}(\sin(\alpha-\beta)+\sin(\alpha+\beta))$ \\
	$\sin^2 \alpha $&$=$&$ \frac12 \left(1-\cos 2\alpha\right)$ &
	$\cos^2 \alpha $&$=$&$ \frac12 \left(1+\cos 2\alpha\right)$ \\
	$\sin^3 \alpha $&$=$&$ \frac14 \left(3\sin \alpha - \sin 3\alpha\right)$ &
	$\cos^3 \alpha $&$=$&$ \frac14 \left(3\cos \alpha + \cos 3\alpha\right)$ \\
	$\frac{\sin 2\alpha}{\sin \alpha} $&$=$&$ 2 \cos \alpha$ &
	$\sin \alpha \cos \alpha $&$=$&$ \frac12 \sin 2\alpha$ \\
	$c^2 $&$=$&$ a^2 + b^2 - 2 a b \cos \gamma$ &
	$\frac{a}{\sin \alpha} = \frac{b}{\sin \beta} $&$=$&$ \frac{c}{\sin \gamma} = 2r = \frac{u}{\pi}$ \\
\end{tabular}
}

\begin{comment}
\begin{center}
	\begin{multicols}{2}
		$\fmm \sin \beta = \frac ba =\frac{\text{Gegenkathete}}{\text{Hypotenuse}}$\\
		$\fmm \cos \beta = \frac ca =\frac{\text{Ankathete}}{\text{Hypotenuse}}$\\
		$\fmm \tan \beta = \frac cb =\frac{\text{Gegenkathete}}{\text{Ankathete}}$\\
		$\fmm \cot \beta = \frac cb =\frac{\text{Ankathete}}{\text{Gegenkathete}}$\\
		\end{multicols}
	\end{center}
$\cos(a+k\cdot2\pi)=\cos(a) \qquad \sin(a+k\cdot2\pi)=\sin(a) \qquad(k \in \mathbb{Z})$\\
$\sin(a \pm b)=\sin(a) \cdot \cos(b) \pm \cos(a) \cdot \sin(b)$\\
$\cos(a \pm b)=\cos(a) \cdot \cos(b) \mp \sin(a) \cdot \sin(b)$\\	
$\tan(a \pm b)=\dfrac{\tan(a) \pm \tan(b)}{1 \mp \tan(a) \cdot \tan(b)}$\\
$\sin(2a)=2\sin(a)\cos(a)$\\
$\cos(2a)=\cos^2(a)-\sin^2(a)=2\cos^2(a)-1=1-2\sin^2(a)$\\

$c^2 = a^2 + b^2 - 2 \cdot a \cdot b \cdot \cos \gamma$\\	
$\frac{a}{\sin \alpha} = \frac{b}{\sin \beta} = \frac{c}{\sin \gamma} = 2r =
		\frac{u}{\pi}$



$\sin(a)\sin(b)=\frac{1}{2}(\cos(a-b)-\cos(a+b))$\\
$\cos(a)\cos(b)=\frac{1}{2}(\cos(a-b)+\cos(a+b))$\\
$\sin(a)\cos(b)=\frac{1}{2}(\sin(a-b)+\sin(a+b))$\\

$\cos^2 \left(\frac{a}{2}\right)=\frac{1+\cos(a)}{2} \qquad
\sin^2 \left(\dfrac{a}{2}\right)=\frac{1-\cos(a)}{2}$\\

$\sin(a)+\sin(b)=2 \cdot \sin \left(\frac{a+b}{2}\right) \cdot
\cos\left(\frac{a-b}{2}\right)$\\
$\sin(a)-\sin(b)=2 \cdot \sin \left(\frac{a-b}{2}\right) \cdot
\cos\left(\frac{a+b}{2}\right)$\\
$\cos(a)+\cos(b)=2 \cdot \cos \left(\frac{a+b}{2}\right) \cdot
\cos\left(\frac{a-b}{2}\right)$\\
$\cos(a)-\cos(b)=-2 \cdot \sin \left(\frac{a+b}{2}\right) \cdot
\sin\left(\frac{a-b}{2}\right)$\\
$\tan(a) \pm \tan(b)=\dfrac{\sin(a \pm b)}{\cos(a)\cos(b)}$\\

$\sin(x) = \frac{1}{2j} \left(e^{jx} - e^{-jx}\right) \qquad
\cos(x) = \frac{1}{2} \left(e^{jx} + e^{-jx}\right)$ \\
$e^{x+jy} = e^x \cdot e^{jy} = e^x \cdot \left(\cos(y) + j\sin(y)\right)$ \\
$e^{j\pi} = e^{-j\pi} = -1$ \\
\end{comment}

\pagebreak
\onecolumn
\includepdf[angle=-90,pagecommand={\section{Periodic Table of the Elements}}, scale=0.9]{periodic_table.pdf}

\end{document}
